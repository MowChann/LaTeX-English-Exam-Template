\documentclass{ExamJHSEngl}
% ExamJHSEngl v0.5.2

\showAnswer{0}

\begin{document}

\chapter[2023]{
  \hspace*{-0.3em}
  \headLine{2023}{年三月包头市昆区三校联考学业水平测试}{英语}{}
}


\section{\Choices{50}}

\setcounter{subsection}{1}
\subsection{完形填空\Scores{15}{1}}
\Directions{阅读短文,从短文后各题所给的四个选项(A、B、C和D)中,选出可以填入空白处的最佳选项,并填写在空白处。}

\setcounter{enumi}{21}
\countercontinue

A traveler was in a large desert, planning to walk across it in one month. Twenty more days passed, the journey had been going on \cloze .“Soon Tl be able to walk out of this desert, ” be thought gladly.

But the desert was never friendly \cloze travelers. In a short time, there came a strong sandstorm. He hurriedly \cloze his head with the clothes, prostrate on the sand. After about ten minutes, the sandstorm \cloze . He shook the clothes and stood up. At that moment, he found himself in a hopeless situation-the backpack with food and water was swept away by the \cloze .

As we know, it seems \cloze to leave the desert without food and water. \cloze , he had a pear left. He held it in his hands tightly. “Not too bad, at least I have a pear. I \cloze I can walk out of the desert.”

Days and nights went by quickly, but the desert still looked endless. Besides, hunger, thirst and fear of \cloze were always around him like ghosts. \cloze , each time he was close to losing hope, he forced himself to stare at the pear that be had been  \cloze . “Not too had, at least I have a pear.”

A small pear became the \cloze for his survival \hint{生存} .Three days later, \cloze he saw a village not far away, he laughed with excitement He felt completely relaxed-the “pear” brought him back to \cloze .

To keep hope is the best weapon \hint{武器} for victory,so never tell you “\cloze” because only if you try to search, you can always find a “pear” to pull you out of trouble.

\begin{enumerate}[start=21,ref={\arabic*},labelsep=-0.1em,itemsep=0em]
  \item[\choice{A}] \options{well}{baddly}{terribly}{heavily}
  \item[\choice{C}] \options{at}{on}{to}{by}
  \item[\choice{D}] \options{played}{beat}{filled}{covered}
  \item[\choice{B}] \options{began}{stopped}{hid}{continued}
  \item[\choice{C}] \options{snow}{desert}{sandstorm}{smoke}
  \item[\choice{B}] \options{right}{impossible}{safe}{important}
  \item[\choice{D}] \options{Carefully}{Easily}{Sadly}{Luckily}
  \item[\choice{A}] \options{believe}{wonder}{remember}{advise}
  \item[\choice{A}] \options{death}{pain}{illness}{sand}
  \item[\choice{D}] \options{Also}{And}{Or}{However}
  \item[\choice{C}] \options{smelling}{eating}{keeping}{imagining}
  \item[\choice{A}] \options{hope}{rule}{question}{plan}
  \item[\choice{D}] \options{as long as}{even if}{so that}{as soon as}
  \item[\choice{C}] \options{danger}{sleep}{life}{mind}
  \item[\choice{B}] \options{anything}{nothing}{something}{everything}
\end{enumerate}


\subsection{阅读理解 \Scores{15}{2}}
\Directions{阅读短文,从每题所给的四个选项(A、B、C和D)中,选出最佳选项,并填写在空白处。}

\ReadingSection{A}

\begin{framed}   

\centerline{\textbf{Car-Free Day}}
\begin{tightcenter}
Life can be so Car-Free.\\
No car, no motorcycle, and no taxi.\\
On Wednesday, June 28\\
from 8 a.m. to 4 p.m. in Zunyi City\\
There are many cities and towns in China for World Car-Free Day.\\
Places like Chongqing are car-free for a day.\\
We invite everyone to the Car-Free celebration to learn more about Car-Free lifestyles.\\ 
\end{tightcenter}
\centerline{\textbf{ Here is the information: }}
\begin{itemize}
  \item Bring a food basket, enjoy live music, and watch the sunset with friends at Zunyi Park.
  \item If you are a student, take your student card to the information desk and you'll get a cup of coffee for free.
  \item Don' t take pets with you.
  \item Take away the rubbish when leaving.
\end{itemize}

\centerline{\textbf{ Enjoy it with us. }}
\end{framed}

\begin{enumerate}[resume,ref={\arabic*},labelsep=-0.1em]

  \item[\choice{A}] How long will the celebration last ?
  \options
    {8 hours.}
    {10 hours.}
    {12 hours.}
    {14 hours.}

  \item[\choice{C}] If Li Lei meets his friend at Zunyi Park on June 28, how will he get there?
  \options
    {By taxi.}
    {By car.}
    {By bike.}
    {By motorcycle.}

  \item[\choice{B}] What can't people do at Zunyi Park?
  \options
    {Listen to music.}
    {Make coffee.}
    {Enjoy the sunset.}
    {Enjoy food.}

\end{enumerate}


\ReadingSection{B}

It's necessary to share your feelings with others. Sharing your feelings can help you to get closer to people who you care about and who care about you.

Sometimes it can be really hard to tell someone that you're feeling sad, worried or upset. But don't keep the feelings locked inside, If that, it can even make you fall sick!

But if you talk with someone who cares about you, like your parents, you will not be all alone with your problems or worries. It doesn't mean your problems and worries go off, but at least someone else knows what your troubles are and helps you to find ways to deal with them.

If a person doesn't understand what you mean right away, try explaining it in a different way or give an example that has something to do with you. Of course, there are many ways for you to talk about your feelings. If you think you'Il have trouble saying what's on your mind, write it down on a piece of paper.

Some children, just like some adults, are more private than others. That means some children will feel shyer about sharing their feelings. A child doesn't have to share every feeling he or she has, but it is important to share feelings when be or she needs help.

\begin{enumerate}[resume,ref={\arabic*},labelsep=-0.1em]

  \item[\choice{C}] What does the underlined sentence tell us?
  \options
    {Don't tell anything to others.}
    {Don't care about what others think}
    {We should share feelings with others.}
    {We have to stay inside and think carefully.}

  \item[\choice{C}] In the writer's opinion, what if you talk with someone about your worries?
  \options
    {They will laugh at you}
    {They will not share the feelings with you.}
    {They will help you find ways.}
    {They will tell others about your worries.}

  \item[\choice{D}] Which of the following is NOT true according to the passage?
  \options
    {Sharing your feelings can make you closer to people.}
    {Worrying too much can even make you fall sick.}
    {There are many ways to talk about your feelings.}
    {Adults are more private than teenagers.}

  \item[\choice{B}] What's the best title for the passage?
  \options
    {How to Make Friends}
    {Sharing Your Feelings with Others}
    {Don't Laugh at Others}
    {Sharing Your Happiness with Others}

\end{enumerate}

\ReadingSection{C}

They say that “travel is the best teacher” and there is no better example of this idea than the Ming Dynasty travel writer and geographer Xu Xiake(1587-1641). His book The Travel Notes of Xu Xiake, not only encouraged a love of travelling among Chinese people but provided important scientific information about the country's land and geography.

Born into a wealthy Jiangyin family, Xu became interested in books about different places at an early age and wanted to travel. When he was 18, however, Xu's father died and so, it seemed, did his travelling dreams. He now was responsible for the family farm and taking care of his 60-year-old mother as tradition required.

But his mother had different ideas. Understanding her son's love of travel and valuing the knowledge he could get from such experiences, this modern-thinking woman refused to keep her son at home. She agreed that Xu could travel for three months every year, when there was less farm work.

So at the age of twenty and with his mother's support, Xu set off for the first time, leaving behind not only his mother but his new wife as well. He would repeat this goodbye each year for most of the next 30 years. During this time, he travelled throughout the Ming kingdom, carefully studying the lands he passed through and recording his experiences and many discoveries in a diary. This diary, which once had over 500,000 words, would eventually become The Travel Notes of Xu Xiake.

Although rich, Xu avoided comfortable travel, preferring to go almost everywhere on foot. This way he could research the environment in detail and get a true picture of the natural world. Many of his trips were to hard-to-reach mountain areas, and through wild forests where few people lived. His willingness to face hardships came at a cost however. Progress was slow and tiring and he was frequently sick, robbed and beaten during his journeys.

Sadly Xu became seriously ill during his last and longest journey, a 4-year trip through the southwest of China. He died in 1641, soon after returning to his hometown for the last time. When his diary was finally printed years after his death, much of it had been destroyed or lost. Although incomplete, it still made Xu a travelling legend around the world.

\begin{enumerate}[resume,ref={\arabic*},labelsep=-0.1em]

  \item[\choice{A}] What is the passage mainly about?
  \options
    {The general details \hint{细节} of Xu Xiake’s life story.}
    {The difficulties Xu Xiake faced in his travels.}
    {The important discoveries made by Xu Xiake.}
    {The influence of Xu's book The Travel Notes of Xu Xiake.}

  \item[\choice{D}] Which of the following best describes Xu's mother? 
  \options
    {Strict but interesting.}
    {Kind but uneducated.}
    {Helpful and hard-working.}
    {Supportive and open-minded. }

  \item[\choice{C}] Why did Xu prefer walking during his travels?
  \options
    {It gave him the chance to meet different kinds of people.}
    {It helped him to save money and travel for a longer time.}
    {It allowed him to see and study the environment in detail.}
    {It was the only way to reach the places he was interested in.}

  \item[\choice{B}] What is the correct order for the following events from Xu's life?

  \begin{itemize}[label={}]
    \item a. He went on his first journey.
    \item b. His book was finally printed.
    \item c. He returned to his hometown for the last time.
    \item d. He developed an interest in books about other places.
    \item e. He started managing the family farm after his father died. 
  \end{itemize}
  \options
    {a-e-d-c-b}
    {d-e-a-c-b}
    {d-e-a-b-c}
    {e-d-a-b-c}

\end{enumerate}

\ReadingSection{D}

A new contact lens \hint{隐形眼镜片} has been developed by scientists from America,Britain and China. It can check for illnesses at any time by examining chemical in tears. A new sensor \hint{感应器} is used to achieve the purpose.

Scientists hope that the improvement could help to deal with sudden medical events. Dr Zhao Yunlong said:“COVID-19 bas had a great influence on the scientific world. And we are doing research to help people deal with similar events in the future.”

When introducing the new contact lens, Dr. Guo Shiqi, a researcher at Harvard, said. “In the past, the sensor was put between the layers \hint{层} of the lens. It didn't have direct contact \hint{接触} with tears. But our new sensor is smaller and smarter. It can be fixed to a lens and keep direct contact with tears. What's more, it causes no discomfort to the eye.”

The contact lens has not been put to the market, but researchers are quite sure that it will be welcomed. It is especially useful for the old and those who may experience sudden medical events.

As society develops, scientists are supposed to develop more products that provide care and convenience \hint{便利} for people. The contact lens is one of what they are researching on. Earlier,a California company developed another kind of contact lens using LED technology. It can show the health information on the mobile phone. Steve Sinclair, head of the company, said,“We have to build something that helps you when you need it and stays off when you don't need it.”

\begin{enumerate}[resume,ref={\arabic*},labelsep=-0.1em]

  \item[\choice{D}] According to Dr. Zhao Yunlong, they do this research because \blank.
  \options
    {they are experienced}
    {they are asked to do so}
    {they want to fight COVID-19}
    {they want to help people }

  \item[\choice{C}] The advantage of the new sensor is that \blank.
  \options
    {it is put between lenses}
    {it s much cheaper}
    {it is smaller and smarter}
    {it can be used for many years }

  \item[\choice{A}] The writer writes the last paragraph by giving \blank.
  \options
    {an opinion and facts}
    {an opinion and answers}
    {a question and facts}
    {a question and answers }

  \item[\choice{B}] The best title of this passage probably is \blank.
  \options
    {Sensors: Science and Technology}
    {Contact Lenses and Convenience}
    {The Most Useful Product}
    {Great Companies of Contact Lens}

\end{enumerate}


\subsection{补全对话(共2节,15分)}
\Directions{从方框内所给的选项中选出能填入空白处的最佳选项,并在答题卡上把该选项涂黑,选项中有两项为多余项。\Scores{5}{1}}

\dialogue[A]{Hey, Stephen.}
\dialogue[B]{Hi! \completion[1.5][c]{D}}
\dialogue[A]{Terrible! I always feel upset.}
\dialogue[B]{What's wrong?}
\dialogue[A]{My parents always ask me to study hard and encourage me before exams. \completion[1.5][c]{E} I don't know how to deal with it}
\dialogue[B]{Don't worry. Why don't you log on to the microblog \hint{登录微博} to share your feelings with others?}
\dialogue[A]{Really? \completion[1.5][c]{B}}
\dialogue[B]{Yes, I think so. \completion[1.5][c]{F} You can get many replies from friends, which may make you feel better.}
\dialogue[A]{\completion[1.5][c]{G}}
\dialogue[B]{Let me see. Well, you'd better have a try. If you have a chance, you shoud suggest your parents read microblog. They will know better.}
\dialogue[A]{Sounds great. Thank you very much.}

\completionchoices
  {Is everything going well?}
  {Is it helpful to me?}
  {They often cook delicious meals for me.}
  {How's it going?}
  {But actually it makes me nervous.}
  {It's very popular among us young people.}
  {But what if my parents don't let me do that?}



\section{\Written{50}}

\setcounter{subsection}{3}
\subsection{补全对话(共2节,15分)}
\Directions{根据下面对话的情景,在每个空白处填入适当的句子,使对话的意义连贯、完整。\Scores{5}{2}}

\dialogue[A]{Good morning, madam. \completion[4]{Can/May I help you}?}
\dialogue[B]{Yes, I'd like to buy a dress, please.}
\dialogue[A]{Oh, good! We've got lots of new dresses in different styles. They were all produced in Shanghai and their quality \hint{质量} is very good.}
\dialogue[B]{Oh, they touch soft. \completion[5]{What are they made of}?}
\dialogue[A]{The blue ones are made of cotton, and those green ones are made of silk.}
\dialogue[B]{But I don't like the colors.}
\dialogue[A]{\completion[5]{What color do you like}?}
\dialogue[B]{Brown.}
\dialogue[A]{Well, this way pleasel Please look at this one. And I think this one is for your age.}
\dialogue[B]{Oh, it's brown. I like brown. Is it made of silk, too?}
\dialogue[A]{Yes. It feels very cool in summer.}
\dialogue[B]{It's good. Can I try it on?}
\dialogue[A]{Of course! Hmm, it fits you very well and you really look beautiful in it.}
\dialogue[B]{\completion[4]{Thank you very much}. How much is it?}
\dialogue[A]{\$78.}
\dialogue[B]{OK. \completion[4]{I'll take it}.}


\subsection{词语运用(共2节,25分)}

\subsubsection{\Scores{10}{1}}
\Directions{用括号中所给单词的适当形式完成句子。(每空仅限1个单词)}

\countercontinue

\begin{enumerate}[start=\value{orders}+1,label={\arabic*.},leftmargin=2em,itemsep=0.75ex]
  \item The novel was written in the \wordform{twenties}, but it still really stands out. (twenty)
  \item The two \wordform{Germans} are our new teachers. They arrived here yesterday. (German)
  \item I am \wordform{terribly} sorry to keep you waiting for a long time. (terrible)
  \item Last summer, I \wordform{milked} the cows and fed the sheep with my grandparents on their farm. (milk)
  \item The public is advised to wear masks when in \wordform{crowded} places. (crowd)
  \item In the electronics market, competition is getting \wordform{hotter} day by day. (hot)
  \item The woman explained clearly again and again to make herself \wordform{understood}. (understand)
  \item It's said that it's possible for them to both improve customers' \wordform{satisfaction} and lower the costs. (satisfy)
  \item Roy saw a small wallet \wordform{lying} in the street on his way back home. (lie)
  \item The math problem \wordform{itself} is too difficult for teenagers to work out by themselves. (it)
\end{enumerate}


\subsubsection{\Scores{10}{1.5}}
\Directions{阅读下面短文,在空白处填入1个适当的单词或括号内单词的正确形式。}

\begin{spacing}{1.5}

Have you ever had a dream to make your home a happier place? Maybe there are some points you can follow.

Sometimes you don't think your parents are fair to you. When you want to dress in \completion[1]{a} modem way, your mum doesn't like you wearing a mini-skirt. When you are making phone calls to friends, they ask \completion{if/whether} you are speaking to a boy or a girl. Sometimes it seems that you're notas close to your parents as you \completion{used} (use) to be. How can you become close again?

Closing the Gap by American wvriter Jay McGraw gives \completion{suggestions} (suggest) on how to have a better relation \completion{with} your parents. He gives ways to help you understand your parents. When you think, “My parents don't want me to have any fun,” that usually means your parents want you to be safe. Both parents and children have needs. They need to feel they are important and \completion{loved} (love). You should tell your parents your needs, and find out what their needs are. Then, you can think of a way to make all of you \completion{happy} (happy). He gives you some ideas:

\begin{itemize}
  \item Make time to talk. You could talk about your school life and your plans for the future.
  \item Keep a diary. This is to help you understand more about \completion{youself} (you) and your feelings.
  \item Show your parents you are \completion{growing} (grow) up. Wash your clothes and help clean the house.
\end{itemize}

Your parents will feel that you are no longer a little child. If you follow these steps, soon you \completion{will be} (be) able to break down the wals between your parents and yourself.

\end{spacing}



\subsection{书面表达(共15分)}

假如你是李华,你的英国好友 Jack 下个学期将来中国学习。下文是他给你发来的邮件,请你根据邮件内容给 Jack 回复一封邮件。邮件应包含箭头所指内容。

\begin{enumerate}[1.]
  \item 词数90左右,开头和结尾已给出,不计入总词数;
  \item 可适当增加细节,以使行文连贯;
  \item 文中不得出现反映考生信息的真实人名、地名等内容。
\end{enumerate}

\begin{minipage}[]{0.6\linewidth}\setlength{\parindent}{1.5em}
    \begin{framed}
\leftline{Hi Li Hua,}
How's it going?\par
Good news! I can study with you in China next term. I can't wait to tell you how happy I am. These days, I am really interested in your daily life both in and out of school. Could you tell me something about it?\par
I know that you are in a volunteer group. It's meaningful. I want to join you to be a volunteer in the coming term. Would you like to tell me what you volunteer to help out?\par
I'm looking forward to your reply.\par
\rightline{Yours,}
\rightline{Jack}
    \end{framed}
\end{minipage}
\begin{minipage}[]{0.05\linewidth}
    $\Rightarrow$
    $\Rightarrow$
\end{minipage}
\begin{minipage}[]{0.3\linewidth}
    \begin{framed}
Favorite subject: ...\\
Favorite teacher: ...\\
Sports: ...\\
Free time activities: ...
    \end{framed}
    \begin{framed}
Endangered animals\\
...
    \end{framed}
\end{minipage}

Hi Jack, 

\ifshowAnswer

I'm so happy that you're coming to study in China next term. \uliner{I'm sure that you'll have fun and learn a lot here.}

\uliner{Our teachers are friendly and patient, especially my English teacher. She often helps me with my
English. So my favorite subject is English. Besides, it's so interesting and useful. I like sports. I usually play soccer or basketball with my friends after school. It's good for health. On weekends, I like to go shopping or watch movies.}

\uliner{Volunteering is great! Welcome to join us. We often work together to clean up the parks, put up
posters for the endangered animals and so on. Seeing what we've helped out, I always feel so proud.}

\else

\begin{spacing}{1.5}
I'm so happy that you're coming to study in China next term. \hlines{8} 
\end{spacing}

\fi

I'm looking forward to meeting you!

\end{document}
